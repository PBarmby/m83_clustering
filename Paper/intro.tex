\section{Introduction}

%\item Galaxies have a lot of discrete sub-components: stars, clusters, nebulae, nucleus.
Galaxies are complex systems, comprised of numerous components with an enourmous range of size,
mass, density, and composition.
These components can be divided into baryonic (stars and their remnants,
nebulae, star clusters, nucleus) and non-baryonic (dark matter);
cataloging the components and describing
the interactions between them is a key step in elucidating the natural history of galaxies.
Only in nearby galaxies can individual sub-components be resolved.
As observational technology has advanced,
the definition of ``nearby'' has changed and will continue to do so, from Milky Way satellites and Local Group galaxies, to a few
Megaparsecs \citep[distance at which stars can be resolved with HST][]{},
to XX Mpc \citep[distance at which stars can be resolved with JWST][]{},
to the entire observable universe with potential future facilities \citep{}.

%\item One way to isolate specific components is with narrow-band filters or CMD analysis.
What is the most efficient way to survey the sub-components of a nearby galaxy?
Here we are discussing components detectable in imaging at ultraviolet through infrared wavelengths,
i.e. with effective temperatures in the range XX--XX~K.
Much cooler or hotter types of objects (molecular gas, accreting compact objects) are better-detected at other wavelengths.
Particular stellar types, or star clusters, are often identified with broad-band colour-magnitude diagrams \citep[e.g.][]{}.
Narrow-band filters can also isolate special stellar types \citep[e.g.][]{} or objects prominent in emission
lines such as planetary nebulae or supernova remnants \citep[e.g.][]{}.
Observations are typically designed with detection of particular classes in mind and sometimes re-used for additional purposes \citep[e.g.][]{}.
Spectroscopic follow-up is often required to confirm candidates.
New observational facilities which provide spatially-resolved spectroscopy  \citep[e.g.]{}{} may reduce the need for separate imaging and follow-up steps,
but greatly increase the complexity of initial data analysis.


Multi-wavelength surveys are extremely common in studies of unresolved galaxies in the distant universe.
While these are often designed to select galaxies or active galactic nuclei with specific properties \citep[e.g.][]{},
sometimes they are pure blank-field surveys.
Broadband 
%($R=\delta \lambda \lesssim X$)
filters are the most common imaging modality,
although there have been a few attempts at narrow- or medium-band surveys as well \citep[e.g.][]{combo-17},
Clustering in colour space can be used to select particular classes of objects from a survey,
for example in selecting AGN via mid-infrared colours \citep[e.g.][]{},
or high-redshift galaxies via Lyman-break dropouts \citep[e.g.][]{}. 
{\bf give some examples here of sophisticated analysis of colour spaces.}

%\item But what if you already have all the filters, and you want to make a census? 
%Can start with properties of known classes of objects \& pick out from multi-dimensional dataset.
%\item Another approach is to see what blind clustering gets you: how many groups and what are they?
%How does this depend on the (number of) wavelengths used?
The purpose of this work is to treat a nearby galaxy as if it were a blank field for surveys, and investigate the
usefulness of different photometric colours for identifying sub-components.
We make use of the Early Release Science (ERS) observations with the Wide-Field Camera 3 (WFC3) of the nearby spiral galaxy M83 \citep{}
and in particular the catalog of point sources produced by \citet{}.
We form colours from the photometric measurements in the catalog and apply several clustering techniques to two-colour datasets.
In conjunction with published catalogs of galaxy components, we identify the optimum parameters for clustering such a photometric dataset,
and the best choices of filter.




%Outline of intro:
%
%\begin{enumerate}
%\item Galaxies have a lot of discrete sub-components: stars, clusters, nebulae, nucleus.
%\item One way to isolate specific components is with narrow-band filters or CMD analysis.
%\item But what if you already have all the filters, and you want to make a census? Can start
%with properties of known classes of objects \& pick out from multi-dimensional dataset.
%\item Another approach is to see what blind clustering gets you: how many groups and what are they?
%How does this depend on the (number of) wavelengths used?
%\end{enumerate}
%
%Work to be cited: 
%\begin{itemize}
%\item astro applications of k-means clustering 
%\item astro applications of other ML techniques
%\item general bkg on galaxy constituents
%\item ??
%\end{itemize}
