Outline of intro:

\begin{enumerate}
\item Galaxies have a lot of discrete sub-components: stars, clusters, nebulae, nucleus.
\item One way to isolate specific components is with narrow-band filters or CMD analysis.
\item But what if you already have all the filters, and you want to make a census? Can start
with properties of known classes of objects \& pick out from multi-dimensional dataset.
\item Another approach is to see what blind clustering gets you: how many groups and what are they?
How does this depend on the (number of) wavelengths used?
\end{enumerate}

Work to be cited: 
\begin{itemize}
\item astro applications of k-means clustering 
\item astro applications of other ML techniques
\item general bkg on galaxy constituents
\item ??
\end{itemize}
