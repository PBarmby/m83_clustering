%\section{Introduction}

% is this too short?

%\item Galaxies have a lot of discrete sub-components: stars, clusters, nebulae, nucleus.
Galaxies are complex systems, comprised of numerous components with an enormous range of size,
mass, density, and composition.
These components can be divided into baryonic (stars and their remnants,
nebulae, star clusters, nucleus) and non-baryonic (dark matter);
cataloging the components and describing the interactions between them is a key step in elucidating the natural history of galaxies.
Only in nearby galaxies can individual sub-components be resolved.
As observational technology has advanced, the definition of ``nearby'' has changed and will continue to do so:
stars can be resolved in Milky Way satellites and Local Group galaxies to distances of about 1~Mpc with the {\em Hubble Space Telescope (HST)},
about 50\% farther with JWST,  reaching the distance of the nearest large elliptical with potential future facilities \citep{brown12}.


%\item One way to isolate specific components is with narrow-band filters or CMD analysis.
What is the most efficient way to survey the sub-components of a nearby galaxy?
Here we are discussing components detectable in imaging at near-infrared through ultraviolet wavelengths,
i.e. with effective temperatures in the range 3000--10000~K.
Much cooler or hotter types of objects (molecular clouds, accreting compact objects) are better-detected at other wavelengths.
Particular stellar types, or star clusters, are often identified with broad-band colour-magnitude diagrams \citep[e.g.][]{chandar10}.
Narrow-band filters can also isolate special stellar types \citep[e.g. Wolf-Rayet stars,][]{hadfield05} or nebulae prominent in emission
lines such as planetary nebulae or supernova remnants \citep{hermann08,blair14}.
Spectroscopic follow-up is often required to confirm candidates.
New observational facilities which provide spatially-resolved spectroscopy \citep[e.g.][]{yan16,sanchez12,drissen10} may reduce the need for separate imaging and follow-up steps,
but greatly increase the complexity of initial data analysis.


Multi-wavelength surveys are extremely common in studies of unresolved galaxies in the distant universe.
While these are often designed to select galaxies or active galactic nuclei (AGN) with specific properties \citep[e.g.][]{trenti11,timlin16},
sometimes they are pure blank-field surveys.
Broadband ($R=\Delta \lambda/\lambda \lesssim 5$) filters are the most common imaging modality,
although there have been some narrow- or medium-band surveys as well \citep[e.g. COMBO-17,][]{combo-17},
Clustering in colour space can be used to select particular classes of objects from a survey,
for example AGN  \citep[e.g.][]{secrest2015, dabrusco09} or extragalactic star clusters \citep[e.g.][]{dabrusco16, hollyhead15}. 

%\item But what if you already have all the filters, and you want to make a census? 
%Can start with properties of known classes of objects \& pick out from multi-dimensional dataset.
%\item Another approach is to see what blind clustering gets you: how many groups and what are they?
%How does this depend on the (number of) wavelengths used?
The purpose of this work is to treat a nearby galaxy as if it were a blind survey, and investigate the
usefulness of different photometric colours for identifying sub-components.
We make use of the Early Release Science (ERS) observations with the Wide-Field Camera 3 (WFC3) of the nearby spiral galaxy M83
and in particular the catalog of point sources produced by \citet{chandar10}.
We form colours from the photometric measurements in the catalog and apply several clustering techniques to multi-colour  datasets.
%In conjunction with published catalogs of galaxy components,
% below is probably a bit strong..
We identify the optimal process for clustering such a photometric dataset and the best choices of filters.


%Outline of intro:
%
%\begin{enumerate}
%\item Galaxies have a lot of discrete sub-components: stars, clusters, nebulae, nucleus.
%\item One way to isolate specific components is with narrow-band filters or CMD analysis.
%\item But what if you already have all the filters, and you want to make a census? Can start
%with properties of known classes of objects \& pick out from multi-dimensional dataset.
%\item Another approach is to see what blind clustering gets you: how many groups and what are they?
%How does this depend on the (number of) wavelengths used?
%\end{enumerate}
%
%Work to be cited: 
%\begin{itemize}
%\item astro applications of k-means clustering 
%\item astro applications of other ML techniques
%\item general bkg on galaxy constituents
%\item ??
%\end{itemize}
