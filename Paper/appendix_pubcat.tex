%Appendix 1 - Published Catalgoues info

As one check on the results of our analysis, we use previously-published identifications of specific types of objects in M83.
We compiled a `published catalog' by combining the contents of the NASA Extragalactic Database (NED) and
[what does it stand for?] \citep[SIMBAD][]{wenger2000} and then adding the catalogs of Wolf-Rayet stars \citep{kim12} and
red supergiant candidates \citep{williams15}, which did not appear in either database.
NED's focus as an extragalactic database and SIMBAD's focus on Galactic objects mean that their contents overlap but are not identical, 
and this is true of the area surrounding M83. A $3\farcm3$ radius region around the coordinates centered at  ($204.26761\deg, -29.839939\deg$)
contains 1553 NED objects and 1772 SIMBAD objects, of which 1220 are matched with each other at 1\arcsec tolerance.
Although the two services use slightly different naming conventions, with human inspection the matches are generally recognizable as referring
to the same object. Interestingly, the databases do not always report the same object type even when the names are identical.
The differences are reasonable in some cases (a supernova remnant can also be an X--ray source, for example), but not others
(e.g. CXOU J133703.0-294945 is reported as a supernova remnant by SIMBAD and an H${\sc ii}$  region by NED).
A detailed study of the databases is beyond the scope of this work; for the purposes of this analysis, we kept the NED classification
for objects which appeared in both databases.
Objects which appeared in one database but not the other were primarily from recent work \citep[e.g.][]{long2014}, from
older studies likely superseded by newer ones \citep[e.g.][]{larsen1999}, or from studies in which only coordinates relative to
the galaxy centre were given \citep{rumstay83,dvpd83}.

Our final combined catalog has 2425 objects of which 750**check** are in the region covered by the ERS catalog.
The main classes are star clusters (350), X--ray sources (105), supernova remnants (86), H${\sc ii}$ regions (81),  and
radio sources (36).
Nearly every entry in the published catalog had an ERS catalog object within 1\arcsec, and the mean distance between
matched objects was 0\farcs26.
Given the nearly 100-fold difference in object density between the two catalogs, matching based on positions alone may 
result in spurious matches **REF**. *Some discussion of the exact matching procedure is warranted here, and a conclusion
on what the best thing to do is.**