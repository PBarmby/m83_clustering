% Abstract
Space-based astronomical observatories generate vast quantities of data.
As technology advances, the size of surveys will grow, and efficient means of analyzing the data they produce are necessary.
Machine learning methods present an effective way to handle large datasets, and are becoming a popular way to analyze large surveys in astronomy. 
The purpose of this research is to apply machine-learning methods to the classification of point sources in the nearby galaxy Messier 83 (M83).
Mean-shift, Affinity Propagation, and K-means, clustering methods were applied to observations of point sources in the M83 galaxy, generated by the Early Release Science Program. 
An object’s light emission over different wavelengths is the key data for classification as it indicates the composition of the object, along with its other physical attributes.
The ERS survey took observations over ten filters in the UVIS channel, from the Wide Field Camera 3 on the Hubble Space Telescope, in the range of Optical to Near Infrared. 
To identify which combination of bands was the best at separating different classes of objects, the strength of the clustering was evaluated and the results compared with independent classification, to determine if each object was correctly identified.
The results of this work will allow astronomers to plan observations that can be used to automatically classify objects in nearby galaxies, leading to more effective surveying, and effecient generation of data.  