% Abstract
Space-based astronomical observatories generate vast quantities of data.
As technology advances, the size of surveys will grow, and efficient means of analyzing the data they produce are necessary.
Machine learning methods present an effective way to handle large datasets, and are becoming a popular way to analyze large surveys in astronomy.
The purpose of this research is to apply machine-learning methods to the classification of point sources in the nearby galaxy Messier 83 (M83).
The Early Release Science Program survey took observations over ten filters in the UVIS channel, from the Wide Field Camera 3 on the Hubble Space Telescope, in the range of Optical to Near Infrared.
Mean-shift, Affinity Propagation, and K-means, clustering methods were applied to observations of point sources in the M83 galaxy.
Colour-colour combinations were created and clustered.
Clustering was performed in two and three dimensions to determine the effectivness of clustering a typical survey.
To identify which colour combination was most effective at separating different classes of objects, the strength of the clustering was evaluated and the results compared with independent classification, to determine if objects were correctly identified.
The most successful combinations are discussed, and a process outlined for the application of these methods to future surveys.
The results of this work will allow astronomers to plan observations that can be used to automatically classify objects in nearby galaxies, leading to more effective surveying, and effecient use of data.