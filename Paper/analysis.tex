% Analysis section
In this section we will outline the process used for each of the clustering methods.
An aim of this work was to help astronomers determine which filters were best at identifying different types of objects in a survey. 
Since the average survey is limited to four filters, different combinations of four filters were used to construct colours for clustering. 
Colours were constructed using the difference between two wavelengths. For a given combination of four filters, all possible colour combinations were used. % Change as needed
The feature space was constructed in two and three dimensions, to add a layer of analysis that isn't typically found in colour-colour space.

\subsection{Clustering Process}
Clustering was performed using all three methods for each colour combination. 
The methods were used independantly and in unison in order to determine the optimal clustering.
The following process allowed the investigation of the effect of all paramters on each clustering technique, leading to an optimal clustering. 
\begin{enumerate}
\item \textbf{Mean-Shift:} Mean-Shift clustering was performed first by estimating the bandwidth paramter with the $estimate_bandwidth$ function in $scikit_learn$.
Following the initial clustering, a "bandwidth hierarchy" was created by performing the clustering again with bandwidth values between $\pm$ intervals of $0.1$ from the estimated bandwidth.

\item \textbf{AP:} Affinity Propagation clustering was then performed by setting the preferences to 10\% of the number of objects in the data set, and setting the damping factor to $0.95$.
Following that clustering, an "Affinity hierarchy" was created by varrying the damping factor, and preference value to determine the effect of each parameter. 

\item \textbf{K-Means:} K-Means clustering was performed last.
The first two clusterings were performed using the number of clusters determined from the initial clusterings by Mean-Shift and AP.
Next, K-Means was performed with $K = \pm 4$ from the original clustering. 
\end{enumerate}
\subsection{Selecting the Clustering}
In order to determine the optimal clustering, a series of metrics were used.
For each clustering, the silhouette score, cluster centers, root mean square, average and standard devation of each colour, and the distances between points in each cluster were calculated. 
\subsubsection{Silhouette Score}
The silhouette score is a metric used to describe the compactness of a cluster in a given clustering and is calculated as an average of all samples in a clustering.  
The silhouette score is given by:
\begin{equation}
\label{eq:ss}
Silhouette Score = \frac{b - a}{\textit{max}\big(a, b\big)}
\end{equation}
where $a$ is the mean intra-cluster distance, and $b$ is the distance between a point and the nearest cluster that point is not a member of.
In addition to the average score of the clustering, the average score for each cluster within the clustering was computed.
The cluster score determines what is driving the average score, and uncovers which clusters are most compact.
