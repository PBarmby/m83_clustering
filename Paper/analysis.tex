% Analysis section
In this section we will outline the process used for each of the clustering methods.
An aim of this work was to help astronomers determine which filters were best at identifying different types of objects in a survey. 
Since the average survey is limited to four filters, different combinations of four filters were used to construct colours for clustering. 
Colours were constructed using the difference between two wavelengths. For a given combination of four filters, all possible colour combinations were used. % Change as needed
The feature space was constructed in two and three dimensions, to add a layer of analysis that isn't typically found in colour-colour space.

Clustering was performed using all three methods for each colour combination. 
The methods were used independantly and in unison in order to determine the optimal object classification.
Mean-Shift and Affinity Propagation were used prior to k-means in order to determine a reasonable number of clusters in the data set.
Once these methods were run, the number of clusters used for k-means input was determined from a combination of Mean-Shift and AP output, and general inspection.

\section{Mean-Shift Clustering}
\section{Affinity Propagation}
\section{K-Means Clustering}
\begin{enumerate}
\item Clustering process
\item Mean-Shift Analysis 
\item Affinity Propagation Analysis 
\item K-Means Analysis
\end{enumerate}