\section{Clustering Analysis}
As the size of galactic surveys grows, the number of dimensions available for analysis increases.
In this survey, 45 different colour combinations are possible, creating a space of 45 possible dimensions. 
Clustering methods provide an efficient way of finding structure in high dimentional data by searching for structure in the feature spaces that cannot be visually inspected. 
A feature space is a set of \textit{n} features that are associated with measurable quantities. 
In this study, each feature space is defined as a set of colours.

\subsection{Princple Component Analysis}
In order to determine the most descriptive features, a Principle Component Analysis (PCA) was performed on each wavelength, and colour. %TODO: Do the PCA



 which can lead to the discovery of new patterns in spaces that we are already comfortable with. 
The methods used in this study are unspervised techniques. Unsupervised clustering takes the data independent of prior classification, and does not rely on prior labeling of the data.

\begin{enumerate}
\item description of clustering and classification 
\item description of PCA
\item description of Mean-Shift
\item description of K-Means
\item experiments with how to apply the techniques
\item final parameters used
\end{enumerate}
