% Results section

\textbf{To be reorganized}
The section will outline the major results of the clustering. 
It is focused on the broad band clustering, and two combinations from the narrow band colours.
The first narrow band combination is the most successful narrow band clustering, and the second is the least successful.
A complete discussion of all the combinations can be found in Appendix 1 \textbf{Make appendix here}.

\subsection{Broad - Broad Band Combinations}
Table~\ref{tab:BBcolours} lists the broad band combinations that were clustered. 
Each combination was clustered in two and three dimensions using K-Means, and Meanshift.
In both dimensions, the $U - B$ and $V - I$ combination was clustered more effectively than the $U - V$ and $B - I$ combination, and will be the combination used for discussion.

\subsubsection{2-Dimensions}
K-Means performed more effectively than Meanshift in 2-dimensions.
K-Means was able to identify the branch of red $U - B$ objects at low values of $K$.
As $K$ increased, the algorithm was also able to identify a group of objects that are red in $V - I$, and blue in $U - B$. 
The optimal clustering selected was at $K=6$ (Figure~\ref{fig:KMBB6} despite the high score of $K=3$.
$K=6$ is the point where the score begins to plateau, and it uncovers more detail in the distribution than $K=3$.

\begin{figure}[H]
\centering
\includegraphics[width=\linewidth]{figs/kmeans_col_6cl_mag05_555-mag05_814vsmag05_336-mag05_438}
\caption{Colour-Colour distribution of the $U - B$ and $V - I$ colours, clustered using K-Means with $K=6$. The colour of each point corresponds to the cluster the point was assigned to. Cluster numbers can be seen in the legend.}
\label{fig:KMBB6}
\end{figure}

K-Means was able to identify objects based on their location in colour space.
The boundaries of each cluster roughly line up with integers in each colour, and identify groups of objects that are relatively blue and red, seperating them from objects with extremes of either colour.

Meanshift was able to identify the branch of objects that is red in the $U - B$ colour (Cluster 2 in Figure~\ref{fig:KMBB6}).
However, the segmentation following that branch did not correspond to meaningful groups in colour space.
Additionally, the bandwidth values caused Meanshift to converge at 5 clusters, where it was unable to identify the branch of red objects.

\subsubsection{3-Dimensions}
Three dimensional clustering was performed with the colour combinations found in Table~\ref{tab:BB3dcolours}.

\begin{table*}
\centering
\caption{Broad band colour combinations in three dimensions}
\label{tab:BBcolours}
\begin{tabular}{lllll}
\hline\hline
Base Colour 1 & Base Colour 2 & Colour 1 & Colour 2 & Colour 3 \\
\hline
$U - B$ & $V - I$ & $U - B$ & $B - V$ & $B - I$ \\
$U - V$ & $B - I$ & $U - V$ & $B - V$ & $V - I$ \\
\hline
\end{tabular}
\end{table*}

Both K-Means and Meanshift identified interesting clusters in three dimensions. 
Similar to two dimensions, K-Means was able to identify objects that were in specific regions of the colour space.
The optimal K-Means clustering was found at $K=4$ (Figure~\ref{fig:BB3dKM4}), as the score peaked, and the red branch of objects was identified.
The three dimensional clustering was driven by the $B - I$ vs. $U - B$ distribution, as the clusters were almost completely seperated in that space.
The clusters overlapped extensively in the $B - V$ vs. $U - B$ space. \textbf{PB: Do you know why this would be? Would it be from the range of colour in the B-I dimension?}
This pattern could be a result of the colour distribution in the $B - I$ space, as the range of object colour in this dimension is much larger than in the $B - V$ colour.

\begin{figure}[H]
\centering
\includegraphics[width=\linewidth]{figs/kmeans_3d_color_4cl_mag05_336-mag05_438vsmag05_438-mag05_555vsmag05_438-mag05_814}
\caption{Colour-Colour-Colour distribution of the $U - B$, $B - V$, and $B - I$ colours, clustered using K-Means with $K=4$. The colour of each point corresponds to the cluster the point was assigned to. Cluster numbers can be seen in the legend.}
\label{fig:BB3dKM4}
\end{figure}

The clustering in three dimensions illustrates how the algorithm is able to identify two distinct branches of objects beyond the dense center of the distribution.
Cluster 4 is a branch of objects that is quite red in both the $U - B$ and $B - V$ colours, but is neutral in the $B - I$ colour.
Cluster 3 is a branch of objects quite blue in the $U - B$ colour, but red in the other two.
The algorithm then segments the dense section of the distribution between the bluer and redder objects in the $B - V$ and $B - I$ colours.
These two clusters are not ideal, as the dense portion of the distribution could be argued as the same cluster. 
However, in order to identify the two distinct branches, this clustering is necessary.
The clusters projected into the base colours can be seen in Figure~\ref{BB2dKM4}, where the branches of objects are clearly identified.

\begin{figure}[H]
\centering
\includegraphics[width=\linewidth]{figs/kmeans_base_color_4cl_mag05_336-mag05_438vsmag05_555-mag05_814}
\caption{Colour-Colour distribution of the $U - B$, and $V - I$ colours, projected from the three dimensional clustering using K-Means with $K=4$. The colour of each point corresponds to the cluster the point was assigned to. Cluster numbers can be seen in the legend.}
\label{fig:BB3dKM4p}
\end{figure}

Meanshift was also able to produce interesting clusters in three dimensions. 
The optimal clustering was chosen at $h=0.75$ which produced 4 clusters.
This clustering was the peak score, but was not the number of clusters for most intervals of bandwidth.
Most bandwidth values predicted 6 clusters, however, the score and cluster seperation at those intervals was quite poor.
Figure~\ref{fig:BB3dMS4} shows the three dimensional clustering at $h=0.75$.

\begin{figure}[H]
\centering
\includegraphics[width=\linewidth]{figs/meanshift_3d_color_4cl_mag05_336-mag05_438vsmag05_438-mag05_555vsmag05_438-mag05_814}
\caption{Colour-Colour-Colour distribution of the $U - B$, $B - V$, and $B - I$ colours, using Meanshift with $h=0.75$. The colour of each point corresponds to the cluster the point was assigned to. Cluster numbers can be seen in the legend.}
\label{fig:BB3dMS4}
\end{figure}

Meanshift was able to identify two groups of objects that sit beyond the dense area of the distribution.
Figure~\ref{fig:BBMS3d4p} shows the projection into the original space, where the cluster location is easier to identify.
The clustering identified two groups of ojbects which are quite red in the $V - I$ colour, but have different $U - B$ colours. 
Cluster 4 is blue in the $U - B$ colour while Cluster 2 is redder.
This identification highlights Meanshift's ability to find outliers in the distribution, as it does not pick out the large branch of objects that are red in $U - B$.

\begin{figure}[H]
\centering
\includegraphics[width=\linewidth]{figs/meanshift_base_color_4cl_mag05_336-mag05_438vsmag05_555-mag05_814}
\caption{Colour-Colour distribution of the $U - B$, and $V - I$ colours, projected from the three dimensional clustering using Meanshift with $h=0.75$. The colour of each point corresponds to the cluster the point was assigned to. Cluster numbers can be seen in the legend.}
\label{fig:BB3dMS4p}
\end{figure}

\subsubsection{M83 Locations}

\paragraph{K-Means Clusterings}
The clusters from the two dimensional clustering show distinct locations in M83.
The branch of red $U - B$ objects found in Cluster 2 of Figure~\ref{fig:KMBB6} are objects that lie loosly around the spiral arms.
These objects are generally not found in the concentrated regions of the arms, and lie to the left and right of these areas.
On the whitelight image, these objects appear to be isolated, dim point sources that could be lying behind clouds of dust, in the back of the galaxy, or on their own outside the arm.
\textbf{PB: What could these objects actually be? Young star clusters, clouds, background sources/galaxies?}

The branch of red $V - I$ objects found in Cluster 1 of Figure~\ref{fig:KMBB6} are objects that lie primarely in the dense regions of the spiral arms, with a few objects lying in the nucleus, and in the region south of the nucleus.
These objects also appear to be quite dim point sources, or no detection in the whitelight image.
This could mean that the objects are some form of cloud, nebula, or background galaxy. 
These objects are interesting as their $V - I$ colour stretches to values over 3, indicating very red emission. \textbf{PB: is this all true...}

Despite the seemingly arbitrary segmentation of clusters 3 and 5, the objects seem to inhabit different regions of M83.
These two clusters generally trace each others location in the galaxy. 
Cluster 5 is generally confined to the denser regions of the arms, while cluster 3 fills in the areas between the objects in cluster 5.
Cluster 5 objects appear to be bright point sources on the whitelight image, while cluster 3 objects are dim or non-existant.
\textbf{PB: what could these objets be?}.

Interestingly, despite their difference in colour, clusters 6 and 4 trace one another around the galaxy similar to clusters 5 and 3.
Cluster 6 objects are concentrated in the centers of the dense regions of the spiral arms, and the intra arm regions.
Cluster 4 objects stay mainly concentrated in the spiral arms, filling in the rest of the dense area between cluster 6 objects and the intra arm regions.
Cluster 4 objects are not as apparent in the intra arm region as cluster 6, but still appear in denser areas. \textbf{PB: what could these objects be?}

In the three dimensional clustering (Figure~\ref{fig:BB3dKM4p}), clusters 4 and 3 follow similar patterns found in two dimensions.
These clusters are the red $U - B$ branch of objects and the red shelf of $V - I$ objects that span the whole range of $U - B$ colour.
Cluster 4 objects are generally located in the spiral arms, but their location is not concentrated, and they are fanned across the entire width of the arms.
This is in agreement with the cluster that identified the red branch in two dimensions, however, three dimensions seems to include more objects that could be considered redder than the rest of the distribution.
Cluster 3 objects trace cluster 4, but are in the dense regions of the spiral arms, and do not fan out. \textbf{PB: What could these be? Should I talk more about the similarities of the 2d and 3d clustering and why one is preferable over the other?}
Clusters 1 and 2 in three dimensions do not provide the same detail as two dimensions as they lump all the objects in the center of the distribution into two groups.
There are no patterns in object location in these two clusters.

\paragraph{Meanshift Clusterings}
The outlying clusters identified by Meanshift in three dimensions are located in interesting areas of M83.
There is almost no overlap between the locations of the objects in these two groups. 
Cluster 2 objects are fanned out through the spiral arms, and in the core of the galaxy.
These objects appear to be dim point sources on the whitelight image, and could be background objects. 
Cluster 4 objects are also found in the spiral arms, however, they are almost only concentrated in the dense areas. 
These objeccts are found in the same regions of the arms as cluster 2, but they are not close to one another. 
This could indicate that these two classes of objects are different physical objects in M83. \textbf{Not sure if this is the right result of their locations}
Lastly, cluster 3, a single object, does not appear to be a meaningful object. It does not seem to be detected in the whitelight image, as it is in an area without a clear, independent point source.
Since this object has very blue colours, it is not clear what it could be, and may be noise. \textbf{Not sure if this is true!}

\subsection{U - OII: Successful Clustering}
The U - OII combination was clustered with the B-V, B-I, and V-I colours using Meanshift followed by KMeans. % More general information about what we are looking for in this combination

\subsubsection{2-Dimensions}

This colour seemed to be much more sensitive to bandwidth selection than other combinations.
With the B-V colour, $h=0.2$ produced $32$ clusters, while $h=0.4$ produced $3$. With the V-I colour, $h=0.35$ produced $17$ clusters, while $h=0.6$ produced $3$.
Due to this sensitivity, the bandwidth hierarchy was created on much narrower increases in $h$, which produced more meaningful clusters.
After producing the narrow hierarchy, the meanshift algorithm predictad a range of clusters from 3 to 13.
In each clustering, the algorithm did not seem to segment the data significantly, as similar to UVW - U, it produced one large cluster with several smaller ones.
The number of clusters predicted reduced linearly with the bandwidth selected, however, the silhouette score saw a sharp drop at $h = 0.33$, which produced 8 clusters, see Figure ~\ref{fig:UOIIMS}. 
This clustering segmented the data into three main groups, which were two "arms" in the distribution that spread to the redder areas of both colours.
Despite picking out these two groups, the two arms contained only approximately 5\% of the data and required further investigation. 

\begin{figure}
\centering
\includegraphics[width=\linewidth]{figs/meanshift_color_8cl_mag05_336-mag05_373vsmag05_438-mag05_555}
\caption{Colour-Colour distribution of the $U-O_{2}$ and B-V colours, clustered using Meanshift with $h=0.33$. The colour of each point corresponds to the cluster the point was assigned to. Cluster numbers can be seen in the legend.}
\label{fig:UOIIMS}
\end{figure}

The K-Means algorithm produced more reliable results, as it produced clusters of relatively similar sizes.
As K increased, the sum of squares value for each clustering decreased, a trend that is expected.
The silhouette score was a maximum at $K=3$, and elbowed at $K=5$. Both clusterings were investigated to determine which was optimal.
At $K=3$, the distribution was segmented according to its U - OII colour.
At $K=5$, the segmentation was similar, however, the section of data that was significantly red in the U-OII colour was given its own cluster.

With each combination of broad bands, the same patterns existed.
This combination in 2-Dimensions did not seem to uncover any more detail or interesting objects than the UVW-U combinations.

\subsubsection{3-Dimensions}

Following the initial clustering, the colours were each broken down into a combination of the OII band and each other band.
The colours used in three dimensions were a combination of U-OII, OII-B, OII-V, and OII-I.

The clustering performance in three dimensions was generally better for almost all clustering parameters.
With all combinations, the clustering algorithms were able to identify a large branch of objects that was fairly red in the U-OII colour, and very blue in the other two, see Figure ~\ref{fig:UOIIKM3d}.
This branch was identified at all values of K, and most values of h.
The added complexity of three dimensions removed the restrictions of only using two dimensions, and allowed the algorithms to cluster the distributions more accurately.

\begin{figure}
\centering
\includegraphics[width=\linewidth]{figs/kmeans_3d_color_3cl_mag05_336-mag05_373vsmag05_373-mag05_438vsmag05_373-mag05_555}
\caption{Colour-Colour distribution of the $U-O_{2}$, $O_{2}-B$, and $O_{2}-V$ colours, clustered using K-Means with $K=3$. The colour of each point corresponds to the cluster the point was assigned to. Cluster numbers can be seen in the legend.}
\label{fig:UOIIKM3d}
\end{figure}

The optimal meanshift clustering was not as apparent in three dimensions.
In the OII - B vs. OII - V combination, the score and number of clusters did not plateau, and the Meanshift clustering was not considered for the optimal clustering.
However, in the OII - B vs. OII - I and OII - V vs. OII - I combinations, the number of clusters elbowed at an h value that maximized the silhouette score.
The number of clusters elbowed at 5, over a range of h values for both combinations.
In each case, the algorithm was able to pick out groups of outliers more clearly than in two dimensions, and the elbow point was taken as the optimal meanshift clustering.

The K-Means algorithm was superior to meanshift for picking out evenly sized groups in all combinations, however, it was not able to pick out some of the detail lying in the groups of outlier data. 
In the combinations that meanshift was successful in, the score peaked at $K=4$, and this was chosen for the optimal clustering. In the last combination, the score peaked at $K=3$.
In addition, each K value was able to pick out the clear branch of objects.
When projected back into two dimensions, the successful segmentation of K-Means can be seen, as it identifies both branches of red objects, and the dense area around zero, see Figure ~\ref{fig:UOIIKM2d}.

\begin{figure}
\centering
\includegraphics[width=\linewidth]{figs/kmeans_base_color_3cl_mag05_336-mag05_373vsmag05_438-mag05_555}
\caption{Colour-Colour distribution of the $U-O_{2}$ and B-V colours, projected from the 3D clustering using K-Means with $K=3$. The colour of each point corresponds to the cluster the point was assigned to. Cluster numbers can be seen in the legend.}
\label{fig:UOIIKM2d}
\end{figure}

\subsubsection{Astronomy Implications}
Both 2 and 3 dimensional clusterings were investigated in the whitelight image. 
The clusterings in three dimensions were able to segment objects more diffinitively than two dimensions.
This was most noticable in the red branches of the distribution.
In two dimensions, the clusters that segmented the red branches seemed to be a combination of dim point sources and objects in the back of the galaxy or behind clouds.
In three dimensions, these clusters were almost entirely objects in the back of the galaxy or behind clouds instead of the combination.
Additionally, the clusters in three dimensions were better able to detect the boundary between the dense center of the distribution and the outlying branches.
When projected onto the galaxy, the objects dense center of the distribution were located in the densist areas of the spiral arms.
The three dimensional clusterings were able to identify these objects and keep them as a seperate cluster.

\subsection{OIII-V: Unsuccessful Clustering}
The $O_{3}$-V colour was clustered with the U-B colour in two dimensions and the U-$O_{3}$, and B-$O_{3}$ colours in three dimensions.

\subsubsection{2-Dimensions}
The K-Means two dimensional clustering segmented the data into sections of U-B colour.
As K increased, K-Means was able to identify a branch of objects that are bluer in both colours.
The segmentation in the colour-colour space translated into the U-B vs B CMD. 
Each clustering segmented the CMD by U-B colour, and the cluster of bluer objects appears to be a group of objects bright in the U band with a brightest B magnitude of approximately 25.
The K-Means score begins to plateau at K=4, however K=5 has a slightly higher score than the rest of the plateau. This is because K=5 is the first clustering to identify the branch of bluer objects.
Despite the plateau, the clustering scores are not as high as other combinations, and the segmentation appears to be arbitrary. 
The Meanshift clustering does not seem to provide meaningful segmentation. The meanshift parameters do not display the same patterns as other combinations. 
The Meanshift score does not plateau at any number of clusters, and the large center cluster contains almost all of the objects in each segmentation.
Meanshift identifies the branch of blue objects at all bandwidth levels, but as the number of clusters increases the clusters are forced into segmenting the blue objects, not the rest of the distribution.
At $h=0.2$, 8 clusters are produced, and Meanshift identifies many clusters in the blue branch, and a larger cluster of objects that are quite red in the U-B cluster.
This clustering results in a peak in the score.
Despite the identification of different parts of the distribution, the algorithms performance does not match the patterns of other combinations, and seems to be a weak colour combination. 

\subsubsection{3-Dimensions}
The three dimensional distribution displays more structure than two dimensions.
Two clear features are visible, a branch of objects that are red in the U-$O_{3}$ colour, and neutral in the rest, and a second branch of objects that are blue in the $O_{3}$-V colour, red in the B-$O_{3}$ colour, and neutral in the U-$O_{3}$ colour.
At all values of K, K-Means is able to identify the first branch of objects. However, it is not until K=6 that the algorithm is able to identify the second branch as its own cluster, see Figure~\ref{fig:OIIIVKM3d}.
By this point, the algorithm has segmented the dense area of the distribution by its U-$O_{3}$ colour.
When projected into two dimensions, there is significant overlap between the clusters that were segmented by colour, and the first branch of objects does not seem to be its own cluster in two dimensions.
The score at K=6 causes a slight peak in the trend, which signifies the effect of picking out both branches of objects. However, the score is still significantly lower than the clusterings that do not identify these branches.

\begin{figure}
\centering
\includegraphics[width=\linewidth]{figs/kmeans_3d_color_6cl_mag05_502-mag05_555vsmag05_336-mag05_502vsmag05_438-mag05_502}
\caption{Colour-Colour-Colour distribution of the U-$O_{3}$, B-$O_{3}$, and $O_{3}$-V colours, clustered using K-Means with $K=6$. The colour of each point corresponds to the cluster the point was assigned to. Cluster numbers can be seen in the legend.}
\label{fig:fig:OIIIVKM3d}
\end{figure}

The Meanshift score plateaued clearly at 7 clusters. There is a large drop in score between 5 and 7 clusters, and both clusterings were able to identify both branches.
Additionally, Meanshift was able to identify sub-clusters within the blue branch of objects, that are objects with extremely blue colours, see Figure ~\ref{fig:OIIIVMS3d}.
The Meanshift clustering was more effective than K-Means, as it did not segment the dense area after it had identified each branch.
The clustering with 5 clusters was chosen as the optimal clustering as the clustering with 7 clusters divided the red branch in two, causing significant overlap in the two and three dimensional spaces.

\begin{figure}
\centering
\includegraphics[width=\linewidth]{figs/meanshift_3d_color_5cl_mag05_502-mag05_555vsmag05_336-mag05_502vsmag05_438-mag05_502}
\caption{Colour-Colour-Colour distribution of the U-$O_{3}$, B-$O_{3}$, and $O_{3}$-V colours, clustered using Meanshift with $h=0.5992$. The colour of each point corresponds to the cluster the point was assigned to. Cluster numbers can be seen in the legend.}
\label{fig:fig:OIIIVMS3d}
\end{figure}

\subsubsection{Astronomy Implications}
After investigating each clustering on the whitelight image, most segmentations did not identify sets of objects that were located in specific areas of the galaxy.
Cluster 4 of the strongest Meanshift clustering identified objects that were located in the less dense regions of the spiral arms of M83.
This cluster isolated the branch of red objects in the colour distribution. Additionally, this cluster was clearly defined in the CMD and split the objects at colour 0.
The largest cluster in the blue branch of objects picked isolated objects in the spiral arm, with only one object located in the nucleus.
All of these objects appear to be background galaxy objects or objects behind clouds, as few of the objects appeared in the whitelight image.
The other two clusters that segmented the blue branch were also objects that appear to be background or covered by clouds, indicating that these objects are quite bright in the $O_{3}$ band, and not in the V band.

