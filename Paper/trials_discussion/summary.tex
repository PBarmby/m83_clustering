\documentclass{article}
\usepackage{graphicx}
\usepackage{sidecap}
\usepackage{epstopdf}

\begin{document}
\title{Summary of Initial Papers} 
\author{Alexander K. Kiar\\Department of Physics \& Astronomy\\Western University\\ \texttt{akiar@uwo.ca}}
\maketitle 

\begin{abstract}
Summary of questions and findings from the eight initial papers from May 27, 2016.
\end{abstract} 

\section{Eight-Dimensional Mid-Infrared/Optical Bayesian Quasar Selection} 
Explored multi-dimensional, multiwavelength selection of quasars from the IRAC and SDSS. 
Selection traditionally in two-colour space, used a combination of 8-D and 4-D techniques. 
Used Bayesian selection techniques and completness and contamination to evaluate selection.  
\begin{enumerate}
\item Converted between Vega and AB photometry
\item IRAC channels: 3.6, 4.5, 5.8, 8.0
\item Made 8 unique colours with ugriz magnitudes 
\item Used all SDSS filters and two short-wave IRAC bands 
\item Bayesian selection section 3
\item Used mean colours to classify types of quasars. Can we use that in our classification? 
\item Set colour limits to reduce error and removed faint and saturated objects. sec 3.1. Can we do the same? 
\item Can we use completeness and contamination? Need a training set, could use a set of points from the data? 
\end{enumerate}

\section{Towards auto classification of all WISE sources}
Applied support vector machines with a training sample to spectroscopic dataset to auto classify objects. 
\begin{enumerate} 
\item used four infrared bands 3.4 - 23 um 
\item used signal to noise 2 in shorter wavelengths and deteriorates in longer. 
\item significant work on colour colour space for WISE survey 
\item used magnitude, color, and differential aperture mag space 
\item computed completeness and contamination for training set. 
\end{enumerate} 

\section{Meaning of WISE colours} 
Colour magnitude criteria to select AGB stars with dust shells and seperate into classes. 
\begin{enumerate} 
\item colour plots showing distribution of object types in survey 
\item set magnitude limits to isolate certain objects. sec 2. 
\item heat map distributions of colors 
\item chose 12 colours, only 3 independent. Used the 3 to classify objects
\item use two-sided Kolmogorov Smirnov test to test distribution hypothesis. Sec 3.3.4
\item created model to predict object type based on colour
\end{enumerate} 

\section{CLaSPS: new method for knowledge extraction} 
Using unspurvised clustering to identify correlations among astronomical obersvations. 
\begin{enumerate} 
\item use combination of features and labels. We have colour features for objects, could we use labels as well? 
\item use a score and fraction of objects similar to our summary. 
\item use Kmeans and vary number of clusters applied to data set
\end{enumerate}

\end{document}