\section{Data}

%\item intro to M83: global parameters (distance, size, environment)
The dataset used for this study is the Wide-Field Camera-3
Early Release Science (ERS) observations of the nearby spiral galaxy Messier 83 (M83).
M83 is a grand-design spiral of type SAB, located at a distance of 4.66~Mpc \citep{tully13}
and the largest member of the M83 subgroup of the nearby Centaurus group of galaxies \citep{tully15}.
The galaxy's apparent radius of $\sim12$~arcmin \citep{} is reasonably well-matched to the camera's field of view (XX true? XX)
{\bf And here we note some other interesting things about M83.}

%\item Intro to WFC3 ERS dataset
The objective of the ERS observations as a whole was to probe star formation in galaxies.
The observations of M83 were made in broad- and narrow-band filters in order to characterize both stellar and nebular properties.
A complete description of the observations and data processing is given by \citet{chandar10}


%\item existing studies with this dataset (cluster, massive stars, etc)
A number of previous studies have used the ERS M83 dataset for various purposes.
These include studies of 
star clusters \citep{chandar10, woffard11, whitmore11, bastian11, bastian12, fousneau12, silva13, andrews14, chandar14, adamo15,ryon15,hollyhead15, sun16},
H~{\sc ii} regions \citep{liu13}, supernova remnants and the interstellar medium \citep{dopita10, hong11, blair14, blair15}, 
resolved stars \citep{kim12, williams15},
and a super-Eddington off-nuclear black hole \citep{soria14}.



%Outline for data section
%%\begin{enumerate}
%\item Intro to WFC3 ERS dataset
%\item intro to M83: global parameters (distance, size, environment)
%\item existing studies with this dataset (cluster, massive stars, etc)
%\item description of catalog (is there a ref for this??)
%\item anything about these data we don't like/didn't use?
%\end{enumerate}
